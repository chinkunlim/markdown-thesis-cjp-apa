\section{關鍵期假說之理論與行為證據}\label{ux95dcux9375ux671fux5047ux8aaaux4e4bux7406ux8ad6ux8207ux884cux70baux8b49ux64da}

關鍵期假說最早由 Lenneberg 首次提出
\autocite{lenneberg1967},他提出語言習得存在一個受生理成熟限制的臨界期。這個臨界期從幼年早期延伸到青春期開始結束,語言習得的能力在青春期(12
或 13
歲)後呈斷崖式下降,此生理基礎與大腦偏側化有關,即青春期大腦偏側化成熟後,語言習得能力將顯著下降。

Johnson 與 Newport 在對 46 位移居美國的華裔與韓裔人士進行語法判斷任務
\autocite{johnson1989},測試他們對語法結構的掌握程度後發現,到達美國的年齡與最終的英語語法能力呈現強負相關,證實了
Lenneberg 提出的語言習得關鍵期假說也適用於第二語言習得(見圖
\ref{fig:johnson})。

\begin{figure}[htbp]
    \centering
    \caption{抵達美國年齡與英語語法測驗總分之關係}\label{fig:johnson}
    \includegraphics[width=0.8\textwidth]{Figure/fig1.png}
    \par\raggedright\footnotesize
    註:圖中顯示受試者在英語語法測驗上的平均總分(Mean Score)隨抵達美國年齡(Age of Arrival)的變化情形。結果顯示,在 3 至 7 歲抵達美國的學習者,其測驗表現與母語人士(Native)無顯著差異;然而,隨著抵達年齡增長(特別是青春期後),測驗分數呈現顯著的線性下降趨勢。資料來源:改編自 Johnson and Newport(1989)。
\end{figure}

Hartshorne 等人在分析約 67
萬人(英語為母語及非母語)的線上語法測驗數據後
\autocite{hartshorne2018},得出的結果強烈支持語言習得的關鍵期假說,且語言習得能力在約
17 歲左右開始劇烈下降。若要達到與母語者相近的水平,學習者必須在 10 到 12
歲之前開始學習英語(見圖 \ref{fig:hartshorne})。

\begin{figure}[htbp]
    \centering
    \caption{語言極致成就與初次接觸年齡之關係}\label{fig:hartshorne}
    \includegraphics[width=0.8\textwidth]{Figure/fig2.png}
    \par\raggedright\footnotesize
    註:本圖顯示單語者(Monolinguals)、沉浸式雙語學習者(Immersion learners)與非沉浸式學習者(Non-immersion learners)的語言極致成就(Ultimate attainment)曲線(經三年浮動窗口平滑化處理)。陰影區域代表 $\pm 1$ 標準誤。數據顯示,若要達到接近母語人士的極致成就,初次接觸英語的年齡必須在 10 至 12 歲之前;此後學習成效開始顯著下滑。資料來源:改編自 Hartshorne et al.(2018)。
\end{figure}

