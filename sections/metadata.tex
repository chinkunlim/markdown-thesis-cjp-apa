% Auto-generated metadata
\newcommand\MyTitleZh{從第二語言習得之關鍵期,探討國小低年級同時授與中文與英文之適切性}
\newcommand\MyTitleEn{Suitability of Simultaneous Chinese and English Instruction in Lower Elementary Grades: A Critical Period Hypothesis Perspective}
\newcommand\MyShortTitle{Critical Period and Bilingual Education}
\newcommand\MyAuthorZh{林錦坤}
\newcommand\MyAuthorEn{Lim, Chin-Kun}
\newcommand\MyAffiliationZh{國立東華大學}
\newcommand\MyAffiliationEn{National Dong Hwa University}
\newcommand\MyKeywordsZh{關鍵期假說、第二語言習得、雙語教育、神經可塑性、跨語言遷移}
\newcommand\MyKeywordsEn{Critical Period Hypothesis, Second Language Acquisition, Bilingual Education, Neuroplasticity, Cross-language Transfer}
\newcommand\MyAbstractZh{在全球化背景下,語言學習的起始年齡一直是教育心理學關注的焦點。本文旨在探討第二語言習得是否存在認知心理學上的關鍵期,並分析國小低年級(二年級以下)同時學習中英文的適切性。透過梳理關鍵期假說(Critical Period Hypothesis)的行為證據與神經認知科學研究(如 fMRI 腦造影證據),本文發現 10 歲前為語言學習的關鍵期,且早期雙語者的大腦傾向利用同一套神經網絡處理雙語。研究結果支持在國小低年級階段同時進行中英文教學,利用孩童神經可塑性的優勢與內隱學習機制,並透過跨語言遷移效應促進語言發展。文末亦針對教學策略提出實務建議。}
\newcommand\MyAbstractEn{In the context of globalization, the optimal age for language learning remains a key focus in educational psychology. This study investigates the validity of the Critical Period Hypothesis in second language acquisition and analyzes the suitability of simultaneous Chinese and English instruction for lower elementary students (Grade 2 and below). By reviewing behavioral evidence and neurocognitive research (e.g., fMRI studies), the findings suggest that the critical period for language acquisition occurs before age 10. Furthermore, early bilinguals tend to process both languages using shared neural networks. The results support simultaneous bilingual instruction in lower elementary grades, leveraging neuroplasticity and implicit learning mechanisms, while facilitating language development through cross-language transfer. Practical teaching strategies are also discussed.}