\section{大腦神經機制之觀點}\label{ux5927ux8166ux795eux7d93ux6a5fux5236ux4e4bux89c0ux9ede}

Kim
等人使用功能性磁振造影(fMRI)研究母語與第二語言在大腦皮層中的表徵是否存在空間分離
\autocite{kim1997}。實驗結果顯示,若第二語言在童年早期習得,母語和第二語言在布洛卡區域中傾向共用相同的皮層區域;反之,在成年後習得第二語言,兩種語言則在鄰近的皮層區域,空間上是分離的(見圖
\ref{fig:kim})。在韋尼克區,無論是童年早期或是成年後習得第二語言,都顯示最小或幾乎沒有空間分離。因此,若在童年早期習得第二語言,大腦能夠利用同一套神經網絡來處理兩種不同的語言。

\begin{figure}[htbp]
    \centering
    \caption{晚期與早期雙語者在布洛卡區(Broca's Area)之語言表徵差異}\label{fig:kim}
    \includegraphics[width=0.8\textwidth]{Figure/fig3.png}
    \par\raggedright\footnotesize
    註:(A)晚期雙語者(Late Bilingual):fMRI 影像顯示母語(紅色)與第二語言(黃色)在下額回(Inferior Frontal Gyrus)的活化區域呈現空間分離(中心點距離 7.9 mm)。(B)早期雙語者(Early Bilingual):影像顯示母語(紅色)與第二語言(黃色)在同一區域呈現高度重疊(橘色,中心點距離 < 1.5 體素),證實早期學習能促使大腦利用同一套神經網絡處理雙語。資料來源:改編自 Kim et al.(1997)。
\end{figure}

Pliatsikas 提出「動態重組模型」(Dynamic Restructuring
Model)\autocite{pliatsikas2020},用時間歷程來解釋在經驗驅動下,神經可塑性如何影響雙語者的大腦結構。雙語經驗能夠促使大腦結構進行適應性調整,增強認知控制與效率,其關鍵在於灰質的階段性轉變。此外,語言學習會導致白質的完整性和連接效率的提高。這種動態的結構能夠幫助大腦對抗衰老過程和病理性神經退化。這表示學習第二語言能夠透過神經可塑性優化大腦結構(見圖
\ref{fig:pliatsikas})。

\begin{figure}[htbp]
    \centering
    \caption{雙語腦的動態重組模型(Dynamic Restructuring Model, DRM)}\label{fig:pliatsikas}
    \includegraphics[width=0.8\textwidth]{Figure/fig4.png}
    \par\raggedright\footnotesize
    註:該模型預測了隨著雙語經驗增加(從初次接觸、固化期到高效期),大腦皮質灰質與白質完整性的動態變化(+代表增加,-代表減少)。模型顯示持續的語言使用能動態優化大腦結構,且白質連結效率在固化期顯著提升。資料來源:改編自 Pliatsikas(2020)。
\end{figure}

