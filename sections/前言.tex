\section{前言}\label{ux524dux8a00}

如今在全球化的背景下,語言不僅是交流與溝通的工具,更有助於獲取新知、增廣見聞。語言學習的合適年齡亦是教育心理學長期關注的焦點之一。根據台灣教育部發布的《十二年國民基本教育課程綱要總綱》\autocite{moe2014},國小英語學科課程從國小三年級開始實施。然而,關於第二語言的習得是否存在認知心理學上的關鍵期?若是,則現行課綱所實施的三年級開始英語課程是否是最佳的介入時機?國小三年級前,學習英語是否會造成語言混淆等負面影響,抑或是更佳的學習黃金時期?本文主要梳理神經認知科學證據,探討關鍵期假說之有效性,並據此分析國小二年級以下同時授予中文與英文的適切性。

