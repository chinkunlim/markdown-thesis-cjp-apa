\section{國小二年級以下小學生同時授與中文與英文之分析}\label{ux570bux5c0fux4e8cux5e74ux7d1aux4ee5ux4e0bux5c0fux5b78ux751fux540cux6642ux6388ux8207ux4e2dux6587ux8207ux82f1ux6587ux4e4bux5206ux6790}

\subsection{介入時機之適切性}\label{ux4ecbux5165ux6642ux6a5fux4e4bux9069ux5207ux6027}

根據 Kim 等人的發現
\autocite{kim1997},越早開始學習第二語言,大腦越傾向將第二語言視為母語的變異體,並利用同一套神經網絡處理。7
歲的孩童學習第二語言,有助於孩童如習得母語般學習英文,而非依賴高認知負荷的翻譯歷程。

Bialystok 指出
\autocite{bialystok2011},當大腦需要同時管理兩種語言系統,並抑制非使用中的語言時,將顯著鍛鍊執行功能(Executive
Functions),特別是抑制控制與注意力轉換的能力(見圖
\ref{fig:bialystok})。

\begin{figure}[htbp]
    \centering
    \caption{單語與雙語兒童在整體-局部任務(Global-local Task)之反應時間差異}\label{fig:bialystok}
    \includegraphics[width=0.8\textwidth]{Figure/fig5.png}
    \par\raggedright\footnotesize
    註:長條圖顯示單語兒童(黑色)與雙語兒童(斜線)在一致(Congruent)與不一致(Incongruent)試驗中的平均反應時間。在需要抑制干擾的不一致情境中,雙語兒童的反應時間顯著快於單語兒童,顯示雙語經驗對執行功能具有正面效益。資料來源:改編自 Bialystok(2011)。
\end{figure}

從認知發展觀點分析,孩童在 7
歲時是學習第二語言的優勢,在於孩童正處於前運思期和具體運思期的過渡,其抽象邏輯思維尚未完全成熟,更傾向於內隱學習。同時,對語音與韻律極為敏感。9
歲的孩童隨著具體運思能力的成熟,逐漸依賴外顯的分析模式學習,反而抑制了對語言的直覺式內隱學習。

綜合上述之論述,我贊成對國小二年級以下的小學生,同時授與中文與英文。現行課綱所規劃之三年級(約
9 歲)開始加入英語學科,雖年齡尚未完全脫離關鍵期,但已逼近 Hartshorne
指出的 10 歲門檻 \autocite{hartshorne2018}。若提早至國小一年級(約 7
歲)介入,則提供了額外兩年的緩衝時期,讓大腦有充裕的時間,在神經可塑性的高峰期建立穩固的連結。

\subsection{跨語言遷移的優勢}\label{ux8de8ux8a9eux8a00ux9077ux79fbux7684ux512aux52e2}

Pasquarella 等人在針對中英雙語兒童的研究發現
\autocite{pasquarella2011},中文的構詞覺識(Morphological
Awareness)能夠正向預測英文的閱讀理解之表現,證實了跨語言遷移(Cross-language
Transfer)的存在。據此研究結果,只要在教學中能夠確保母語的穩固發展,中文和英文可以形成互補的學習優勢(見圖
\ref{fig:pasquarella})。

\begin{figure}[htbp]
    \centering
    \caption{中英雙語兒童形態覺識之跨語言結構方程模型}\label{fig:pasquarella}
    \includegraphics[width=0.8\textwidth]{Figure/fig6.png}
    \par\raggedright\footnotesize
    註:路徑分析顯示各語言變項間的預測關係。結果顯示「英文複合詞覺識」(English Compound Awareness)與「中文詞彙」(Chinese Vocabulary)之間存在顯著的雙向預測關係(標準化係數分別為 .30 與 .31),證實了跨語言遷移(Cross-language Transfer)的存在。資料來源:改編自 Pasquarella et al.(2011)。
\end{figure}

\subsection{教學策略之實務建議}\label{ux6559ux5b78ux7b56ux7565ux4e4bux5be6ux52d9ux5efaux8b70}

雖然神經科學研究結果支持早期介入,但在實務上仍須謹慎防範學習第二語言時犧牲了母語的發展,避免中英文都未精熟的現象,故教學策略顯得至關重要。同時,國小低年級的學生的具體運思能力尚未成熟,應避免同時授予中文和英文的文法規則,並利用大腦對語音的敏感度,側重於聽和說的內隱學習。

語言習得依賴大量的可理解輸入。依據現行課綱的節數安排,英語課程在第二學習階段(小三、小四)每週也僅規劃
1
節課,這種少次數的安排僅能視為語言接觸,難以滿足建立專屬神經連結所需的刺激門檻。因此,在低年級同時授予中文與英文的實務上,可參考臺北市的實施模式
\autocite{taipei2024},善用課綱中的彈性學習課程,採用內容與語言整合學習,即將英語融入低年級的操作性課程中,增加接觸時數,彌補課綱所規定的節數之不足。此方式亦與林子斌提出的台灣雙語教育的本土模式之觀點相互呼應
\autocite{lin2020},即在非英語系國家的情境下,營造普及與友善的雙語環境,讓學生在生活情境中自然習得語言。

